\documentclass[11pt,a4paper,sans,english]{article}

%enabling umlaute
\usepackage[utf8]{inputenc}

%page geometry
%\usepackage[a4paper,top=3cm,bottom=3cm,left=2.5cm,right=1.5cm]{geometry}
\usepackage[a4paper,top=1cm,bottom=1cm,left=2cm,right=2cm]{geometry}
%\usepackage[a5paper,top=1cm,bottom=1cm,left=2cm,right=2cm]{geometry}

\usepackage[english,german]{babel}

%Mathematik
\usepackage{amsmath}
%Bildverwaltung
\usepackage{graphicx}

\usepackage{float}
\usepackage{hyperref}




\begin{document}
\section{Introduction and general considerations}
\subsection{Time considerations}
Thailand is one of the countries in which there is a rainy or monsoon season. 
The rainy season usually takes place between June and August but this depends on the specific regions. The north and south of Thailand can have different durations. In the north you can expect somewhere between 15 and 20 days of rain with 220mm of rain.
I would recommend checking where you want to go before deciding on a time period.
Note that some regions are claimed to be more enjoyable in the rainy season such as Isan in the north-east.
Some information on the rainy season can be found \href{https://thaiest.com/thailand/weather/rainy-season#:~:text=The\%20months\%20of\%20June\%20to,the\%20water\%20during\%20this\%20time.}{here}.

We have been to Thailand during the month of Juli for two weeks and have been quite lucky with the rains. We have visited Chiang Mai (4 days) and Chiang Rai (2 days) in the north, Krabi (4 days) in the south and Bangkok (2 days) before we left.
A direct flight from Zürich takes roughly 10 hours and you will move forward in time by 5 hours.
We planned a complete day for each transit, which was quite a good idea even when the transit only is a flight of 1h (you still need to check in baggage and get to and from your accomodation!).

When deciding on a timeframe, you should consider that Thailand is a big country and has lots to see. The northern and southern parts are quite different and are each worth a visit.
\subsection{Flora and fauna}
Thailand is in a tropical monsoon climate and has quite a varied flora. Depending on the region you might find tropical rainforests, palm trees or forests made up of birch, chestnuts amongst others.
Due to its varied flora, you will also find quite a varied fauna. You might encounter various kinds of lizards (e.g. monitor lizards), monkeys, elephants. There will also be quite a lot of insects, with quite a lot of mosquitos.
\subsection{Food and drinks to look out for}
Food:
\begin{itemize}
	\itemsep-0.5em 
	\item Pad Thai
	\item Pad Krapau (quite spicy)
	\item Tom Kha Gai
	\item Khao Soi (curry with fried noodles)
	\item Khantoke (lots of small dishes)
\end{itemize}
Drinks:
\begin{itemize}
	\itemsep-0.5em 
	\item Fresh smoothies!
	\item Iced teas (Thai, green)
\end{itemize}
\subsection{Financial considerations}
The flight to Thailand is currently still quite expensive (roughly 1500ChF per person both ways) but the food, accomodation and tours are fairly cheap. Expensive dishes are around 300 Baht (about 7.5ChF) with street food costing around 100 Baht (2.5ChF). Accomodation for us was around 200ChF for 2 persons for 3 days, though we regularly did not go for expensive options. The tours cost around 1000 to 2000 Baht (25-50ChF) per person, depending on what and how long you're choosing. 
\section{Chiang Mai}
\subsection{General information}
The province of Chiang Mai has lots of small city and only one big city: Chiang Mai. 
Things to do in Chiang Mai
\begin{itemize}
	\itemsep-0.5em 
	\item Go to Doi Inthanon (highest mountain in Thailand, part of the Himalaya)
	\item Go to see Elephants (please consider choosing an ethical sanctuary. Generally speaking less interaction is more "natural" and riding often requires the handlers to be quite strict and use an elephant hook. We have been to the \href{https://www.elephantdreamproject.com/{elephant dream project}})
	\item Hiking
	\item Trekking in the jungle
	\item Visiting all the temples
	\item Visiting indigineous people (hill tribes such as the Karen)
	\item Night markets
\end{itemize}
While restaurants were quite nice in Chiang Mai, personally I found small streetfood shops at the night markets most enjoyable. Chiang Mai is relatively small and well traversable by foot (if you enjoy walking). Given the left side driving and the "interesting" driving style of the locals I would recommend not booking a car.
\subsection{Our experience}
We stayed in \href{https://www.booking.com/hotel/th/kristi-house-echiiyngaihm1.de.html}{Kristi House}, which is relatively small without breakfast, though there are options for breakfast nearby. In Kristi House there is a tour operator with who we booked a day trip to Doi Inthanon and another day trip to an elephant sanctuary.
Both trips took up roughly a whole day (pickup at the hotel around 7-8 and dropoff at the hotel around 18). 
\subsection{Doi Inthanon} 
Our guide stopped at several places along the way, like a waterfall, the highest point of Thailand, a Hmong market. Before going on a trek with a Karen person through the forest (we saw small snakes!) we stopped for lunch.
I would recommend this tour, if you like seeing the flora and fauna in the jungly mountains in the north and enjoy trekking.
\subsection{Elephant dream project}
The sanctuary is located between 1.5 and 2h outside of Chiang Mai at the Doi Inthanon. We visited a group of 4 elephants, were allowed to feed bananas and grass to the elephants, had a quick walk up the hill to see the elephants feeding at a secondary location and were able to join the elephants at the waterplace (where they sometimes, but not always swim). If you book a half day tour, you will leave after lunch, if you book the full day tour, you can harvest and feed some grass for the elephants and make and feed a digestion supplement out of bananas, corn husks, rice and other ingredients.
\subsection{Chiang Mai city}
In Chiang Mai there are lots and lots of temples to see and visit. While they are all quite similar, each one has a slightly different style and different art inside. We have tried to visit as many temples as possible. We also have visited multple night markets and strolled around them. We mostly had dinner at one of them.

\section{Chiang Rai}
\subsection{General information}
The province of Chiang Rai has lots of small city and only one big city: Chiang Rai. 
Things to do on Big Island:
\begin{itemize}
	\itemsep-0.5em 
	\item Visit temples (white, blue)
	\item Trekking in jungle
	\item Hiking
	\item Visit golden Triangle
\end{itemize}
\subsection{Our experience}
We stayed in \href{https://www.booking.com/hotel/th/saikaew-resort.de.html?aid=318615&label=German_Switzerland_DE_CH_29561940745-4La_x1JtjPk4ljDcius9eQS640874808148\%3Apl\%3Ata\%3Ap1\%3Ap2\%3Aac\%3Aap\%3Aneg\%3Afi55583925958\%3Atidsa-302083110424\%3Alp1003277\%3Ali\%3Adec\%3Adm\%3Aag29561940745\%3Acmp313804345&sid=ef02e173088c39a4d4cfe3124886fffe&checkin=2023-10-21;checkout=2023-10-22;dest_id=-3247118;dest_type=city;dist=0;group_adults=4;group_children=0;hapos=1;hpos=1;no_rooms=2;req_adults=4;req_children=0;room1=A\%2CA;room2=A\%2CA;sb_price_type=total;soh=1;sr_order=popularity;srepoch=1691079435;srpvid=e10272814ca60095;type=total;ucfs=1&#no_availability_msg}{small lovely place} at the outskirts of the city (40min by foot to the city center). They have bikes you can use for free. With the hotel we booked a day tour to visit all the temples and the golden triangle and another one for trekking in the jungle.
\subsection{Trekking in the jungle} 
Trekking in the jungle was a special experience. Trekking in 30-35°C weather in the jungle with 90\%+ humidity is quite exhausting. The trek itself was not too technical though a sturdy pair of hiking boots is recommended. You should remember to bring mosquito repellent, sun screen as well as hats against the sun. Our guide cooked food in bamboo for lunch (chicken, omelette and rice) which is a unique experience.
\subsection{Temple and golden triangle tour}
The temple and golden triangle tour is a packed day (start around 8, finish around 19) but well worth it. We visited many different temples, the most impressive of which being the white temple; were brought to the monkey cave, where two rivaling gangs of monkeys live freely. Our guide fed them bringing them relatively close to our group, with no incident occuring. After a lunch break we drove to the Birma/Myanmar border and visited the golden triangle (Myanmar, Thailand, Laos), where we visited an opium museum. 
\subsection{Chiang Rai city}
In Chiang Rai there are some temples and restaurants with not too much to see. The night market was quite nice with lots of stand and street food. In Chiang Rai we have had Khantoke which is northern Thai dish the owners of Rebstock recommend us to try. It consists of lots of small dishes such as sausages as well as other kinds of meat and dips. 
\section{Krabi}

\subsection{General information}
Krabi is in the south of Thailand and is renowned as a resort area. 

\subsection{Our experience}
We have stayed in an
\href{https://www.booking.com/hotel/th/anana-ecological-resort-krabi.html?checkin=2023-07-16&req_children=0&from_sn=android&type=total&no_rooms=1&checkout=2023-07-20&keep_landing=1&sb_price_type=total&aid=2090015&sid=acf2fc5ef45a08118dd3eb8ae140615c&activeTab=main&req_adults=2&label=Share-Do7gHL\%25401684087118-vtudyES\%401685808487&auth_success=1&group_adults=2&dist=0&group_children=0&room1=A\%2CA#_}{ecological resort} which grows its own food. While it was quite rainy when we were there, it was still quite nice, due to the well furnished rooms and the nice hotel restaurant. In roughly 300m distance from the entrance there is also a small restaurant operated by two nice ladies. We can recommend this. Due to the relatively bad weather, we only stayed at the pool when it wasn't too rainy and didn't get to do an island tour. If we would have had the opportunity we would have gone for a tour to see different islands including Koh Phi Phi.

\section{Bangkok}
Bangkok is surprisingly large with lots of things to see and do. We initially planned only 2.5 days to see Bangkok but did not see everything there is to see. That said, 2.5 days is enough to get a rough understanding of Bangkok.
From Bangkok you could do a one day trip to Ayutthaya which we meant to do but chose not to do in the end (quite exhausted at the end of the trip).

\subsection{Our experience}
We have stayed in an
\href{https://www.booking.com/hotel/th/the-yard-hostel.html?aid=2090015&label=Share-Do7gHL\%25401684087118-vtudyES\%401685808487&sid=ef02e173088c39a4d4cfe3124886fffe&all_sr_blocks=131612903_362126892_0_1_0\%2C131612903_362126892_0_1_0\%2C131612903_362126892_0_1_0\%2C131612903_362126892_0_1_0;checkin=2023-10-21;checkout=2023-10-22;dest_id=-3414440;dest_type=city;dist=0;group_adults=4;group_children=0;hapos=1;highlighted_blocks=131612903_362126892_0_1_0\%2C131612903_362126892_0_1_0\%2C131612903_362126892_0_1_0\%2C131612903_362126892_0_1_0;hpos=1;matching_block_id=131612903_362126892_0_1_0;no_rooms=2;req_adults=4;req_children=0;room1=A\%2CA;room2=A\%2CA;sb_price_type=total;sr_order=popularity;sr_pri_blocks=131612903_362126892_0_1_0__46811\%2C131612903_362126892_0_1_0__46811\%2C131612903_362126892_0_1_0__46811\%2C131612903_362126892_0_1_0__46811;srepoch=1691081632;srpvid=787b76cd8ab90357;type=total;ucfs=1&#hotelTmpl}{container hostel} in a small street just of the main road. It was a nice experience though the shades do little against light in the morning and the rains can be quite loud.
We have done two self organised trips to the \href{https://www.chatuchakmarket.org/}{Chatuchak weekend market} and to china town.
\subsection{China Town}
China town is quite expansive with lots and lots of small backstreets with stands. It is worth a trip as it has a quite distinct atmosphere and feels like you are in a different country. If you don't like small markets and strolling around, china town might not be for you.
\subsection{Chatuchak weekend market}
The Chatuchak weekend market is one of, if not the biggest weekend market in the world. According to their homepage there are 15000 stands and you need 3h to see a bit of everything and 5h if you want to actually try to haggle for stuff. We have spend roughly 4h on the market and feel like we saw roughly half of what is to see. It is a wet market, meaning there are live animals (dogs, cats, mice, hamster, snakes, spiders, chameleons, monkeys, ...) for sale. If you enjoy weekend markets it is well worth a trip!
\section{Other things to see}
\begin{itemize}
	\item Ayutthaya
	\item Sukhothai
	\item Lots of islands
	\item Isan
	\item Phuket (known as the party province, roughly comparable to Mallorca)
\end{itemize}
There is likely way more to see, but I cannot remember them right now.

\end{document}