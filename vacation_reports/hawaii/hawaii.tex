\documentclass[11pt,a4paper,sans,english]{article}

%enabling umlaute
\usepackage[utf8]{inputenc}

%page geometry
%\usepackage[a4paper,top=3cm,bottom=3cm,left=2.5cm,right=1.5cm]{geometry}
\usepackage[a4paper,top=1cm,bottom=1cm,left=2cm,right=2cm]{geometry}
%\usepackage[a5paper,top=1cm,bottom=1cm,left=2cm,right=2cm]{geometry}

\usepackage[english,german]{babel}

%Mathematik
\usepackage{amsmath}
%Bildverwaltung
\usepackage{graphicx}

\usepackage{float}
\usepackage{hyperref}




\begin{document}
\section{Introduction and general considerations}
\subsection{Time considerations}
We have been to Big Island, Oahu and Maui for 16 days in total, so roughly 5 days on each island. Depending on how much you want to pack in each day I would maybe recommend even 7 days per island. The trip from Europe takes door to door roughly 24 hours in a single timezone and the time shift including the flights will be +12 hours from Europe to Hawai'i and +36 from Hawai'i to Europe. 
\subsection{Flora and fauna}
Hawai'i prides itself on its endemic species and reefs and therefore only coral reef sunscreens are allowed. I have not seen this enforced but would strongly recommend using them. They are mineral oil based and can be bought on the islands.

Generally, there are tight restrictions on the things you can bring in (and out) of Hawai'i to ensure the safety of the flora and fauna
\subsection{Food and drinks to look out for}
Food:
\begin{itemize}
	\itemsep-0.5em 
	\item Luau
	\item Ahi (yellow fin tuna)
	\item Acai bowls (acai based ice cream + fruit and granola)
	\item Kalua pig
	\item Fresh fruit
	\item basically any fresh fish
	\item Loco moco (it's a lot of food, I'm not sure this is a proper recommendation)
\end{itemize}
Drinks:
\begin{itemize}
	\itemsep-0.5em 
	\item Mai Tai
	\item Lava flow
	\item Coffee grown on Hawai'i (Kona coffee is known to be quite nice. When buying this, you need to check that its 100\% Kona coffee!)
	\item basically any coconut or pineapple based drink
\end{itemize}
\subsection{Financial considerations}
Hawai'i is a relatively expensive place. Compared to other places everything is expensive: the (transatlantic) flights, the hotels/ resorts, the food and the cost for local activities.
That being said, for me it was worth the money.
\section{Maui}
\subsection{General information}
`Big' cities on Maui:
\begin{itemize}
	\itemsep-0.5em 
	\item Kahului
	\item Hana
	\item Lahaina
	\item Kihei
\end{itemize}
Things to do on Maui:
\begin{itemize}
	\itemsep-0.5em 
	\item Road to Hana
	\item Haleakala
	\item Snorkeling
	\item Hiking
	\item Beaches
	\item Camping
	\item Botanical gardens (didn't do that)
	\item Maui Chocolate Coffee Tours (didn't do that)
\end{itemize}
Short list of nice places to eat or drink
\begin{itemize}
	\itemsep-0.5em 
	\item \href{https://flatbreadcompany.com/}{flat bread company} in Paia. They sell basically Hawaiian Pizza with special topings.
	\item \href{https://www.dthmaui.com/lunch-dinner-menu/}{down the hatch} in Lahaina
	\item \href{https://www.google.com/maps/place/Nutcharee's+Authentic+Thai+Food/@20.7489383,-156.4577975,19z/data=!4m10!1m3!2m2!1sRestaurants!6e5!3m5!1s0x7954d041fd44637b:0x745d4c979ca5c0b3!8m2!3d20.7489383!4d-156.4567032!16s%2Fg%2F11bz_2lp07}{Nutcharee's Authentic Thai Food} in Kihei
\end{itemize}
Given the size of Maui, I would generally recommend booking a car.
\subsection{Our experience}
We have been on tour on Maui and were camping in different spots of the island. If you want to do this, it is important to book the camping sites well in advance. Camping at the side of the road is in principle possible but not allowed and can result in hefty fines (we were told in the area of 500\$).

We have booked the car with all camping equipment via AirBnB (\href{https://www.airbnb.ch/rooms/616735436873706871?locale=de&_set_bev_on_new_domain=1662911427_MTY3OGU4ZDE4YjEz&source_impression_id=p3_1679528299_XFqNRPO05C1SrsGq}{Epic Maui Camping}; also available on \href{https://www.booking.com/hotel/us/epic-maui-camping-1.de.html?auth_success=1}{booking}) and can recommend this. There were some parts missing on the first day because the car was not packed properly but the hosts fixed this as soon as possible. They gave a lot of recommendations on what to do, where to eat and drink and such.

\subsection{Camping}
During our camping trip we have slept two nights (split) at da beehive (we got the contact from our host) against a donation, one night in Hana (Carolines, we got the contact from our host) and two nights in Camp Olowalu (you \underline{need} to book well in advance!). All locations have been quite lovely but were also quite varied. 

Da beehive is located in the country side with the closest town at least 20 minutes by car away, but has a beautiful view of the ocean and sky. It is also quite close to Haleakala and can be used as a starting point for a visit there. There is no flowing water there.

Carolines in Hana is a private camping ground at the edge of Hana. This was well organised with your own designated spot on the camping ground and a small shower installed. There are several beaches close by. Hana is on the windward and therefore wetter part of the island.

Camp Olowalu is a organised camping ground run by the national park services of the US on the leeward and therefore drier part of the island. There are different options available: a spot to park your care and camp there (we chose this one), a camping spot (might be better as it is less cramped) and a cabin. There is a long beach directly at the camp. The camp has flowing water and shower stalls but during the midday the water supply is turned off. There are also feral chickens running around the camp. Depending how deep your sleep is, they will wake you up.

\subsection{Activities}
\subsubsection{Haleakala}
To enter the national park you need a pass. You can either buy a pass only for Haleakala or also for other parks on Hawai'i such as the Vulcanoes park on Big Island (worth a visit).

You can drive up the Haleakala (roughly at 3k altitude) on a well built winding road. It is an inactive vulcano and has very rare flowers and insects. Going off the path is strongly advised against (and maybe even forbidden). There is the option to camp on Haleakala on a camping ground (book \underline{well!} in advance) or in a cabin in the `wilderness area'. We have done neither.
You can watch the sunrise and sunset. For sunrise you need to reserve a spot for you (also well in advance, there are a few spots left to book the day before), while for sunset you can always go. It can get quite crowded at sunset. If you want to park the car at the top of the mountain you should arrive at least 1 hour before sunset.

\subsubsection{Road to Hana} 
A roughly 34 mile long road going from traveling from the more leeward side of the island to the windward side of the island. You will see lots and lots of waterfalls and quite a bit of tropical jungle. There are several spots where you can park the car and do small hikes or see waterfalls. Some interesting spots were the \href{https://www.google.com/maps/place/Twin+Falls+Maui+Waterfall/@20.911966,-156.243829,13z/data=!4m20!1m13!4m12!1m6!1m2!1s0x7954ac25622d34d3:0xfa34511da051c993!2sH%C4%81na,+Hawaii+96713,+USA!2m2!1d-155.9879885!2d20.7557169!1m3!2m2!1d-156.2660229!2d20.9115514!3e0!3m5!1s0x7eab357b40b6d0ed:0x51beb222d42815b3!8m2!3d20.9113614!4d-156.2439879!16s%2Fg%2F11cm_3x94j}{Twin Falls}, the bamboo forest and the \href{https://www.google.com/maps/place/Waikamoi+Nature+Trailhead+and+Parking+Area/@20.8755849,-156.1867716,15z/data=!4m20!1m13!4m12!1m6!1m2!1s0x7954ac25622d34d3:0xfa34511da051c993!2sH%C4%81na,+Hawaii+96713,+USA!2m2!1d-155.9879885!2d20.7557169!1m3!2m2!1d-156.2660229!2d20.9115514!3e0!3m5!1s0x7eab4b290067919f:0xca99a7f8e011c0c0!8m2!3d20.8755788!4d-156.1867851!16s%2Fg%2F1tcwrbwy}{Waikamoi Nature Trailhead}.
For me the road to Hana is a \textbf{must} on Maui.

\subsubsection{Snorkeling}
We have been snorkeling with \href{https://mauisnorkelcharters.com/}{maui snorkeling charters} at Molokini crater. There are lots of endemic fish to be seen as well as octupus (if you are lucky, they are hiding quite well) and \href{https://www.holualoainn.com/the-hawaiian-honu-symbol-of-wisdom-and-good-luck/}{Hawaiian Honu}, the green seaturtle. I would recommend stowing away all jewelry in a safe place (e.g. on the boat). 

\section{Big Island}
\subsection{General information}
`Big' cities on Big Island:
\begin{itemize}
	\itemsep-0.5em 
	\item Kailua-Kona
	\item Hilo
\end{itemize}
Things to do on Big Island:
\begin{itemize}
	\itemsep-0.5em 
	\item Vulcanoes park
	\item Snorkeling
	\item Hiking
	\item Beaches
	\item Glamping
	\item Botanical gardens
	\item Manta Ray night dive (didn't do that, but would want to)
	\item Going up Mauna Kea or Mauna Loa (didn't do that, but would want to)
\end{itemize}
Given the size of Big Island, I would generally recommend booking a car, especially if you want to go to vulcanoes park.
\subsection{Our experience}
On Big Island we have been on both sides of Big Island and stayed in Kona and close to Hilo. After the camping trip on Maui, we took it slow and didn't plan too much apart from visiting the Vulcanoes park.
\subsection{Sleeping places}
There are some resorts you might want to stay in (e.g. north of Kona) and some nice hotels you could book. We chose to go for a few days of \href{https://www.booking.com/hotel/us/hamakua-house-campign-cabanas.de.html}{glamping} close to Hilo and then at in an \href{https://www.airbnb.ch/trips/v1/ac98664b-0f96-471e-a81e-2f1417b230bb/ro/RESERVATION_USER_CHECKIN/HM5HQT5QFF}{airbnb} in Kona. The airbnb wasn't anything special, but since we were away most of the time it was perfect and the hosts were quite nice. 

You might want to consider staying in the vulcanoes region if you want to spend a lot of time in the vulcanoes park.

In the Kona area there are coffee farms and some offer that you can sleep there. When we tried to book they already had no places left.
\subsection{Activities}
\subsubsection{Visit to vulcanoes park}
You can visit the vulcanoes park. To enter it you must have a ticket (the combined pass with Haleakala is good for a year). In the park itself there is a nice looking hotel with a restaurant, which we didn't try. We went there for a full day of hiking and honestly you can hike in that park for multiple days and still have lots to see. A park ranger recommended a drive in the confines of the park which is roughly 4 hours round trip (by car!) without too much stopping.
You should consider watching the caldea when its dark (before sunrise or after sunset).

The vulcanoes park can be hazardous if you are reckless. You should stay on the trails, if you don't you might break through the earth and drop into hot steam vents (roughly 100$^\circ$C). This can cause serious burns and is a cause of death for multiple people a year.

Visiting the vulcanoes park is a must for me.
\subsubsection{Waterfalls around Hilo}
In the Hilo area, there are several water falls that can be visited. We have been to Akaka falls and Rainbow falls. 

For Rainbow falls you should try to get there before noon, as at that time there is a good chance to see rainbows in the splash of the waterfall. After noon the angle of the sunlight is incorrect and this is then impossible. However, there is not much to see at the rainbow falls, just the water fall and no trail.

Akaka falls has a short trail (roughly 3km round trip) and is one of the longest free falling water falls. This is a nice short trip.
\subsubsection{Kona}
Around Kona we did not travel too much and explored the city a bit as well as going to different beaches and doing the Kona Brewing brewery tour. The tour is roughly 15 to 20 minutes tour with information and then roughly half an hour in the brewery lounge with beer tasting. 
\section{Oahu}

\subsection{General information}
`Big' cities on Oahu:
\begin{itemize}
	\itemsep-0.5em 
	\item Honululu
\end{itemize}
Things to do on Oahu:
\begin{itemize}
	\itemsep-0.5em 
	\item Polynesian cultural center
	\item Snorkeling
	\item Hiking
	\item Beaches
	\item Ala moana shopping center (biggest open air world wide)
	\item See a hula dance
	\item Pearl harbour (didn't go there)
\end{itemize}
On Oahu there is a relatively well connected public transport system (you have to pay in cash and there is no return money), you don't need to book a car.
\subsection{Our experience}
Since the conference took place in Honululu, we started and ended our trip on Honululu. I would not recommend this very much.
In Honululu we stayed in Hotels but friends stayed in a place in the north of the island where a group of turtles happened to come by every evening.
Given that I just arrived on Hawai'i, I recovered from the jet lag the first few days enjoying nice acai bowls, going to the beach in Honululu and doing some short hikes. One of them was to Diamond head, a vulcano crater that is quite nice to hike in. On the second stay on Oahu we visited the polynesian cultural center and were at the beach again.

There are also other places on Oahu to visit and stay in, but given our schedule we decided to stay in Honululu.
\subsection{Sleeping places}
We have stayed in \href{https://www.booking.com/hotel/us/castle-queen-kapiolani.de.html?aid=318615&label=New_German_DE_CH_20153971705-S6%2ALWXwOFxWcOJzS8j2k7gS634186709656%3Apl%3Ata%3Ap1%3Ap2%3Aac%3Aap%3Aneg%3Afi%3Atiaud-297601666995%3Adsa-64415224945%3Alp20126%3Ali%3Adec%3Adm%3Aag20153971705%3Acmp313804345&sid=a15b87db5efa027e9b71c5d5e5ac0b18&all_sr_blocks=18062436_91911162_2_0_0;checkin=2023-05-16;checkout=2023-05-19;dest_id=20030916;dest_type=city;dist=0;group_adults=2;group_children=0;hapos=1;highlighted_blocks=18062436_91911162_2_0_0;hpos=1;matching_block_id=18062436_91911162_2_0_0;no_rooms=1;req_adults=2;req_children=0;room1=A%2CA;sb_price_type=total;sr_order=popularity;sr_pri_blocks=18062436_91911162_2_0_0__76555;srepoch=1679604129;srpvid=c9c2918fe6a50003;type=total;ucfs=1&#hotelTmpl}{Queen Kapiolani Hotel} which was quite nice and in the middle of Honululu. On the second leg we stayed in the airport hotel, which was also quite nice.
\subsection{Activities}
\subsubsection{Honululu}
Honululu has quite a few beaches you can visit, lots of restaurants, nice bars and all of these things. Honululu has many different districts, some very "asian" parts (e.g. Chinatown) or more "american" parts (e.g. Ala moana or waikiki).
\subsubsection{Hiking}
We have been walking around Honululu, i.e. mostly Ala moana and Waikiki. The biggest hike we did was to Diamond head. This has a quite nice view.

While we did not do too many other hikes, there are quite some nice ones out there, e.g. \href{https://www.google.com/maps/place/Wiliwilinui+Ridge+Trail/@21.2990018,-157.7626496,3a,75y,90t/data=!3m8!1e2!3m6!1sAF1QipOX2D0XrLsWhbfhpLxErdWDhM3vmblX6s3Yt1KS!2e10!3e12!6shttps:%2F%2Flh5.googleusercontent.com%2Fp%2FAF1QipOX2D0XrLsWhbfhpLxErdWDhM3vmblX6s3Yt1KS%3Dw152-h86-k-no!7i4000!8i2252!4m16!1m8!3m7!1s0x7bffdb064f79e005:0x4b7782d274cc8628!2sHawaii,+USA!3b1!8m2!3d19.8967662!4d-155.5827818!16zL20vMDNnaDQ!3m6!1s0x7c006d34a758a5d1:0x5a7f2802644997d2!8m2!3d21.2990018!4d-157.7626496!10e5!16s%2Fg%2F11bxdz3_5n}{Wiliwilinui Ridge Trail}.
\subsubsection{Polynesian cultural center}
The \href{https://polynesia.com/villages/}{polynesian cultural center} (PCC) is a park that shows and teaches you a bit of polynesian culture. There are "islands" for Hawai'i, Tahiti, Samoa, Aotearoa, Fiji and Tonga. If you want to see every single one, you need to probably go more than a single day. You can do different activities like fire knife twirling or see presentations like a haka. Everything we did or see was very nice and worth the time.

At the end of the day there are two events: the HA Breath of life show and a luau. Neither is covered in the base package of the ticket.
The Ha Breath of life show is basically a musical. For each of the islands there is a small section displaying their kind of music, culture and dances. This was really nice.
The luau is a dinner with the lots of polynesian food such as kaula pig that roasts roughly 8 hours on that day and lots of fresh food.

To enter the PCC you need to buy a ticket. The ticket is relatively expensive but covers three days of entry. You can also book a shuttle ride to different hotels in Waikiki. There are different packages covering "only" a self guided tour without the HA Breath of life show, up to a privately guided tour with the best seating at the show and the luau. 

I can really recommend going there.
In the end we decided to go for a group guided tour which was quite nice since you did not need to think about what to do and could ask lots and lots of questions. 

\section{Kauai}
We did not have the time to visit Kauai, but it is said to be a more wild and less touristic island. Some scenes of Jurassic Park have been filmed there. Activities are more on the hiking side of things.
\end{document}
